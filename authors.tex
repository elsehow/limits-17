\author{Nick Merrill}
\affiliation{%
  \institution{BioSENSE, UC Berkeley School of Information}
  \city{Berkeley} 
  \state{California, USA} 
}
\email{ffff@berkeley.edu}

\hypersetup{
 pdfauthor={},
 pdftitle={},
 pdfkeywords={},
 pdfsubject={},
 pdfcreator={Emacs 25.1.1 (Org mode 9.0.4)}, 
 pdflang={English}}

\begin{abstract}
In February of 2017, Google announced the  first SHA1 collision.
Using over nine quintillion computations (over 6,500 years of compute time),
a group of academic and industry researchers produced two different PDF files with identical SHA1 checksums.
But why? After all, SHA1 had already been deprecated by numerous standards and advisory bodies.
This paper uses the SHA1 collision compute as a site for surfacing the space of ecological risks, and sociotechnical rewards, associated with the performance of large computes.
I forward a theory of polemic computation, in which the feat of expending material (ecological, labor, and opportunity) costs,
rather than particular computational results,
exert agency in particular sociotechnical discourses.
This paper does not make specific claims about the (ecological, political, labor) limits within which polemic computes must operate in order to be considered acceptable.
Instead, this paper raises the question of how such limits could be established, in the face of polemic computes' significant costs and difficult-to-measure rewards.
\end{abstract}


\begin{CCSXML}
<ccs2012>
<concept>
<concept_id>10010405.10010476</concept_id>
<concept_desc>Applied computing~Computers in other domains</concept_desc>
<concept_significance>500</concept_significance>
</concept>
<concept>
<concept_id>10003120.10003121.10003126</concept_id>
<concept_desc>Human-centered computing~HCI theory, concepts and models</concept_desc>
<concept_significance>300</concept_significance>
</concept>
</ccs2012>
\end{CCSXML}

\ccsdesc[500]{Applied computing~Computers in other domains}
\ccsdesc[300]{Human-centered computing~HCI theory, concepts and models}

% We no longer use \terms command
%\terms{Theory}

\keywords{theory, limits, polemics, charisma}

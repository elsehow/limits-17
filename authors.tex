\author{Nick Merrill}
\affiliation{%
  \institution{BioSENSE, UC Berkeley School of Information}
  \city{Berkeley} 
  \state{California, USA} 
}
\email{ffff@berkeley.edu}

\hypersetup{
 pdfauthor={},
 pdftitle={},
 pdfkeywords={},
 pdfsubject={},
 pdfcreator={Emacs 25.1.1 (Org mode 9.0.4)}, 
 pdflang={English}}

\begin{abstract}
In February of 2017, Google announced the  rst SHA1 collision. Using over nine quintillion computations (over 6,500 years of com- pute time), a group of academic and industry researchers produced two di erent PDF  les with identical SHA1 checksums. But why? After all, SHA1 had already been deprecated by numerous regu- latory and advisory bodies. This paper uses the SHA1 collision compute as a site for surfacing the space of ecological risks, and sociotechnical rewards, associated with the performance of large computes. I forward a theory of polemic computation, in which the feat of expending material (ecological, labor, and opportunity) costs exert agency in particular sociotechnical discourses. I conclude with challenges for future work in balancing the costs and rewards of computes with polemic aims.
\end{abstract}


\begin{CCSXML}
<ccs2012>
<concept>
<concept_id>10010405.10010476</concept_id>
<concept_desc>Applied computing~Computers in other domains</concept_desc>
<concept_significance>500</concept_significance>
</concept>
<concept>
<concept_id>10003120.10003121.10003126</concept_id>
<concept_desc>Human-centered computing~HCI theory, concepts and models</concept_desc>
<concept_significance>300</concept_significance>
</concept>
</ccs2012>
\end{CCSXML}

\ccsdesc[500]{Applied computing~Computers in other domains}
\ccsdesc[300]{Human-centered computing~HCI theory, concepts and models}

% We no longer use \terms command
%\terms{Theory}

\keywords{theory, limits, polemics, charisma}
